\section{Related Work}

Pharmacovigilance is the pharmacological science which deals with drug safety. It studies
the collection, detection, monitoring, assessment, and prevention of harmful effects with
pharmaceutical products. Mining social media messages such as tweets, articles, and Facebook
posts for health and drug related information has received significant interest in
pharmacovigilance research since people tend to share their daily activities on social media.
Since pharmacovigilance heavily focuses on adverse drug reactions, it necessitates an automatic
detection of the personal intake of medicine by sensing the social media to build smart
health-care applications. Thus, developing automated classification models for identifying
messages (e.g., tweets) containing description of personal intake of medicine is a pragmatic
step towards the automation of Pharmacovigilance. Cramer~\emph{et al.}~\cite{cramer1989often} 
found that as we increase the number of dosages of medicines (say from one to four times) in 
a day, compliance rate decreases. Thus, several research work~\cite{harmark2008pharmacovigilance, saparova2012motivating, cambria2012sentic} 
discussed and explored different healthcare problems though advanced technologies. 
In this section, we provide a brief literature review on this problem.

Neural networks technologies have yielded immense success in computer vision, natural 
language processing, speech processing, machine translation, and 
healthcare~\cite{kim2014convolutional, liang2014deep, shin2016lexicon, denkowski2017stronger}. 
However, training deep neural networks to obtain good models is not easy and depends on 
determining hyper-parameters optimally~\cite{glorot2010understanding}. 
Bergstra~\emph{et al.}~\cite{bergstra2012random} performed the random search for hyper-parameter 
optimization. Moreover, Lim~\emph{et al.}~\cite{lin2014user} used deep neural network 
techniques to detect user-level psychological stress from social media. However, only
going deeper with convolutions does not lead to the best solution~\cite{szegedy2015going}. 
Thus, recent studies such as Le~\emph{et al.}~\cite{le2017convolutional} explored 
shallow networks for text classification. They found that their shallow word models outperform 
deep models. This encouraged us to use shallow networks in our proposed approach 
to identify personal medication intake from Twitter.

Due to an active participation of users on social media, it is now feasible
to classify latent user attributes (\emph{e.g.}, gender, age, regional origin,
and political orientation) in social media such as Twitter~\cite{rao2010classifying}.
Research studies in last a few years indicate that social media is heavily used
in building healthcare applications~\cite{grajales2014social, toxicovigilance, rosenthal2017semeval}.
Sarker~\emph{et al.}~\cite{sarker2015utilizing} presented a review of pharmacovigilance techniques from
social media data and discussed a possible pathway for automated pharmacovigilance research.
Recently, Klein~\emph{et al.}~\cite{klein2017detecting} presented an annotated corpus
to train machine learning models to find whether a tweet with a medication implies
that the person (i.e, Twitterer) has taken that medication at any specific time.

Often one solution to a complex problem does not fit to all scenarios~\cite{bell2010all}.
Thus, researchers use ensemble techniques to address such problems. For instance, 
Wang~\emph{et al.}~\cite{wang2008biomedical} presented classifiers ensemble approaches 
for biomedical named entity recognition by combining combining Generalized Winnow, 
Conditional Random Fields, Support Vector Machine, and Maximum Entropy through three 
different strategies. Moreover, ensemble techniques have shown to perform well in 
biomedical entity extraction~\cite{ekbal2013stacked} and named entity 
recognition~\cite{sikdar2012differential, speck2014ensemble}. Furthermore, 
stacked ensemble techniques are very useful in different healthcare 
applications~\cite{dinakar2014stacked, speck2014ensemble}.

Our literature review confirms that leveraging social media data using ensemble and 
neural network techniques is very beneficial in healthcare applications. 
Zhou~\emph{et al.}~\cite{zhou2002ensembling} presented a neural network ensemble and 
proposed that many neural networks can be jointly used to solve a problem efficiently. 
For instance, Deng and Platt~\cite{deng2014ensemble} ensemble deep learning for speech 
recognition. Moreover, a recent work~\cite{liu2016ensemble} proposed several classifier 
ensembles to effectively distinguish between adverse drug events (ADEs) and non-ADEs in 
informal text on social media. Thus, in order to solve the problem of identifying personal 
medication intake from Twitter, we train a stacked ensemble of shallow convolutional neural 
network (CNN) models on an annotated dataset (see Section~\ref{evaluation}). We use random 
search for tuning the hyper-parameters of the CNN and build an ensemble of best models that 
achieved the best performance. 


